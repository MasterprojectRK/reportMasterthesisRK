\section*{Abstract}

In recent years, many studies have shown that the three-dimensional conformation of genomes is a key factor
for understanding several important mechanisms on the molecular biological level.
However, the so-called Hi-C experiments typically conducted to measure this structure are still costly,
so that computational methods for predicting the spatial chromatin organization from existing data have recently gained some attention. 

In this thesis, two machine learning approaches are investigated with 
regard to their usability for predicting chromosome conformation in form of Hi-C contact matrices from so-called ChIP-seq data.
Here, the first method adapts and extends an existing dense neural network architecture for Hi-C matrix predictions, 
while the novel second method, Hi-cGAN, leverages techniques from image synthesis, especially conditional generative adversarial networks (cGANs).

While the dense neural network approach can neither produce satisfactory predictions for the Hi\hbox{-}C matrices of human cell lines GM12878 and K562,
nor for Drosophila Melanogaster embryonal cells in the chosen setting, Hi-cGAN yields encouraging outcomes in all three cases.


\section*{Zusammenfassung}

In den letzten Jahren wurde in mehreren Studien gezeigt, dass die dreidimensionale Struktur von Genomen ein
wichtiger Schlüssel für das Verständnis zahlreicher wichtiger Vorgänge auf molekularbiologischer Ebene ist.
Das sogenannte Hi-C Verfahren, welches zum Erfassen dieser 3D-Struktur typischerweise angewandt wird,
ist derzeit jedoch noch recht aufwendig, sodass computergestützte Verfahren zur Vorhersage 
der räumlichen Struktur aus existierenden Daten an Bedeutung gewonnen haben.

In dieser Masterarbeit werden zwei auf Techniken des maschinellen Lernens beruhende Methoden im Hinblick auf ihre Eignung  
zur Vorhersage von Hi-C Kontaktmatrizen aus sogenannten ChIP-seq Daten untersucht.
Während die erste Methode auf einem existierenden, voll verbundenen neuronalen Netzwerk für die Vorhersage von Hi-C Matrizen aufbaut,
handelt es sich bei der zweiten Methode, Hi-cGAN, um einen neuartigen Ansatz, der sich Techniken aus der maschinellen Bild-Synthese zu Nutzen macht,
insbesondere sogenannte ``conditional generative adversarial'' Netzwerke (cGANs).

Während das voll verbundene neuronale Netz im gewählten Aufbau weder für die menschlichen Zell-Linien GM12878 und K562, 
noch für embryonale Zellen der Fruchtfliege Drosophila Melanogaster zuriedenstellende Vorhersagen der jeweiligen Hi-C Kontaktmatrizen treffen kann, 
führt Hi-cGAN in allen drei Fällen zu viel versprechenden Resultaten.
