\section{Method details}
\subsection{Input data}
\begin{itemize}
 \item source of data
 \item how it has been processed
\end{itemize}

\subsubsection{Hi-C matrices} \label{sec:methods:hicMatrices}
\begin{itemize}
 \item GEO numbers
 \item processing with cooler coarsen
\end{itemize}
Contrary to the paper, which is using a distance-aware normalization, 
these matrices have not been normalized or scaled for the thesis at hand
due to bad experiences during the previous study project \cite{Krauth2020}.

\subsubsection{ChIP-seq data}
\begin{itemize}
 \item ENCODE as source
 \item Scaling to [0...1] for processing
 \item refrain from whitening, etc. -- the data is no image with inherent correlations
\end{itemize}

\subsection{Dense Neural Network approach} \label{sec:methods:denseNN}
\subsubsection{Basic setup}
\begin{itemize}
 \item nr neurons
 \item activations
 \item initialization
 \item optimizer, learning rate
 \item refrain from mirroring, little benefit for doubling input sample numbers
\end{itemize}

\subsubsection{Modifying kernel size, number of filter layers and filters}
maybe not needed
\subsubsection{Modifying input resolution}
maybe not needed

\subsubsection{Custom loss function based on TAD insulation score}
\begin{itemize}
 \item computation details, custom layer
 \item full insulation score takes too long, simplified
\end{itemize}

\subsubsection{Combination of mean squared error, perception loss and TV loss}
maybe not needed

\subsection{HiC-GAN approach}
\subsubsection{Using a DNN for 1D-2D conversion}
\subsubsection{Using a pre-trained DNN for 1D-2D conversion}
\subsubsection{Using a CNN for 1D-2D conversion}








