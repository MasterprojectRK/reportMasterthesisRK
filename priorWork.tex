\section{Related work}
In the last five years, several approaches have been presented to determine DNA-DNA interactions \emph{in silico}, 
using existing data from various experiments. \autoref{sec:prior:predictingInteractions} gives an overview about these methods.
Furthermore, some methods originally developed for image synthesis and similar tasks in computer vision might also be useful 
in the field of Hi-C matrix generation and are thus summarized in \autoref{sec:prior:generativeCV}.

\subsection{Methods for predicting DNA-DNA interactions and contact matrices} \label{sec:prior:predictingInteractions}
As of 2020, there is quite a body of existing work in the field of predicting DNA-DNA interactions, 
using various approaches and different types of input data.

Two conceptually similar methods have been proposed by Brackley et al. in 2016 and MacPherson et al. in 2018 \cite{Brackley2016, MacPherson2018}.
In both approaches, DNA is modeled as a ``beads-on-a-string'' polymer, and simulation techniques are employed to
find energy-optimal spatial structures of these polymers.
Apart from constraints derived from the molecule's DNA sequence itself, the models also consider spatial contact constraints derived from \acrshort{cs} experiments
of chromatin factors which are known to mediate such DNA-DNA contacts.
The interaction matrices derived from the simulations look interesting, 
but the paper from Brackley et al. \cite{Brackley2016} is unfortunately lacking a comparison with ``true'' experimentally measured Hi-C matrices, 
and the results from MacPherson et al. \cite{MacPherson2018} seem inferior to most other ones presented in this section.

Another simulation-based method has been developed by di Pierro et al. in 2017 \cite{Pierro2017} and later extended by Qi and Zhang \cite{Qi2019}. 
In both cases, a \acrfull{cnn} is trained to learn different ``open'' and ``closed'' chromatin states from 11 chromatin factors, 
and the predicted chromatin states are then taken as constraints for beads-on-a-string models.
The difference between \cite{Pierro2017} and \cite{Qi2019} lies mainly in the number of states considered and the simulation methods applied;
the results are mathematically convincing in both cases.

A further approach using chromatin states is due to Farr\`e and Emberly \cite{Farre2018}.
Here, the conditional probability of two genomic regions being in contact, given their distance and the chromatin state around them,
is estimated using Bayes' rule. 
In this case, the chromatin state -- reduced to active or inactive -- is derived from \acrfull{damid} signals of 53 chromatin factors using probabilistic methods \cite{Zhou2016}.
The conditional probabilities on the right side of Bayes' rule are either computed from training data or estimated with different probabilistic approaches, too.
While the predicted contact matrices do not look like real Hi-C matrices with this approach, highly interacting regions are still often well identifiable.




Conceptually probably the most simple method in this regard is HiC-Reg, 
proposed by Zhang et. al in 2019~\cite{Zhang2019}. 
It is a machine learning approach, using random forest regression to predict interactions
between two genomic ``windows'', using \acrshort{cs} data of 13 chromatin features, i.\,e. transcription factors and histone modifications, as well
as DNase-seq data and the genomic distance of the two windows. 
The published results for five human cell lines look interesting, but could not be reproduced in two study projects at the university of Freiburg~\cite{Krauth2020,Bajorat2019}.

Another machine-learning approach for direct prediction of interaction matrices was presented by Farr\`e et al. \cite{Farre2018a}
in 2018. Here, a one-dimensional convolutional filter is used to convert \acrshort{cs} data from 50 chromatin factors into a 
one-dimensional chromatin vector, which is then processed by a \acrfull{dnn} to predict interaction matrices. 
The weights of the network are trained on experimental, distance-normalized Hi-C matrices.
The published results for drosophila melanogaster chromosomes look again interesting, since the predicted matrices resemble the general structure of real matrices quite well.



\subsection{Generative techniques from computer vision} \label{sec:prior:generativeCV}
