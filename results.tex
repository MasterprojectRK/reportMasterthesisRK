\section{Results}
\subsection{Dense Neural Network approaches} 

\subsubsection{Initial results for comparison} \label{sec:initialDNNresults}
The basic network was trained for 3000 epochs as explained in \xxx.
The validation error (MSE) for the basic neural network reached a minimum of about \SI{150000}{} after 500 epochs, \cref{fig:results:basicDNN_lossEpochs}.
Beyond that, the learning curve indicates overfitting, but differences between predictions at 500 and 1000 epochs are hard to spot in the matrix plots from the test set, 
\cref{fig:results:00500_19_030-040} and \ref{fig:results:01000_19_030-040}.
\begin{figure}[hbp]
 \centering
 \scriptsize
 \import{figures/basic_dnn_results/}{lossOverEpochs.pdf_tex}
 \caption{learning progress for basic DNN} \label{fig:results:basicDNN_lossEpochs}
\end{figure}
To asses the quality of the predictions, distance-stratified Pearson correlations were computed between predicted and real matrices,
up to the maximum distance from the diagonal given by windowsize and binsize, i\,e. for distances up to $80\cdot \SI{25}{\kilo\bp}=\SI{2}{\mega\bp}$ in the basic setting.
While the suitability of Pearson correlation as metric for seems debatable for Hi-C matrices \xxx, it is quite common in the field and thus useful at least for comparisons.
\Cref{fig:results:basicDNN_pearson} shows the Pearson correlations alongside area under the correlation curve (AUC) for the five test chromosomes at hand.
The red lines in each correlation plot show the correlation between the corresponding training matrix from GM12878 and the target matrix from K562.
It is obvious that all predicted test matrices have a strictly positive Pearson correlation, but are not better than transferring data from the training cell line.

The predicted matrices themselves looked modest when plotted with pygenometracks \xxx. 
While some of the highly interacting regions, for example between \xxx and \xxx of chromosome 19 were
well predicted, other structures, especially larger ones like the one between \xxx and \xxx in chromosome 19 
or between \xxx and \xxx in chromosome 21 are not predicted at all.

\begin{figure}[p]
    \begin{subfigure}{0.45\textwidth}
        \scriptsize
        \resizebox{\textwidth}{!}{
        \import{figures/basic_dnn_results/}{pearson_chr03.pdf_tex}}
        \caption{chr3}
    \end{subfigure} \hfill
    \begin{subfigure}{0.45\textwidth}
        \scriptsize
        \resizebox{\textwidth}{!}{
        \import{figures/basic_dnn_results/}{pearson_chr05.pdf_tex}}
        \caption{chr5}
    \end{subfigure}\\[5mm]
    \begin{subfigure}{0.45\textwidth}
        \scriptsize
        \resizebox{\textwidth}{!}{
        \import{figures/basic_dnn_results/}{pearson_chr10.pdf_tex}}
        \caption{chr10}
    \end{subfigure}\hfill
    \begin{subfigure}{0.45\textwidth}
        \scriptsize
        \resizebox{\textwidth}{!}{
        \import{figures/basic_dnn_results/}{pearson_chr19.pdf_tex}}
        \caption{chr19}
    \end{subfigure}\\[3mm]
    \centering
    \begin{subfigure}{0.45\textwidth}
        \scriptsize
        \resizebox{\textwidth}{!}{
        \import{figures/basic_dnn_results/}{pearson_chr21.pdf_tex}}
        \caption{chr21}
    \end{subfigure}
    \caption{Pearson correlations, basic DNN, test chromosomes}
    \label{fig:results:basicDNN_pearson}
\end{figure}

\begin{figure}[p]
    \scriptsize
    \import{figures/basic_dnn_results/}{pred00500_chr21_030-040.pdf_tex}
    \caption{example prediction, 500 epochs} \label{fig:results:00500_21_030-040}
\end{figure}
\begin{figure}[p]
    \scriptsize
    \import{figures/basic_dnn_results/}{pred00500_chr19_030-040.pdf_tex}
    \caption{example prediction, 500 epochs} \label{fig:results:00500_19_030-040}
\end{figure}
\begin{figure}[p]
    \scriptsize
    \import{figures/basic_dnn_results/}{pred01000_chr19_030-040.pdf_tex}
    \caption{example prediction, 1000 epochs} \label{fig:results:01000_19_030-040}
\end{figure}

\subsubsection{Results for variations of the convolutional part}
The predictions from the ``wider'' variant were generally similar to the initial results,
both in terms of Pearson correlations and in terms of matrix plots, \cref{fig:results:widerDNN_pearson} and \xxx.
Given the small increase in the number of trainable parameters at overall similar network topology, this is not surprising.
However, it is interesting that hardly any improvement was found.
Overfitting was less obvious than with the initial setup and the training process looked more smooth overall, 
but the remaining validation error was slightly higher than for the initial approach, \cref{fig:results:widerDNN_lossEpochs}.
\begin{figure}[p]
    \begin{subfigure}{0.45\textwidth}
        \scriptsize
        \resizebox{\textwidth}{!}{
        \import{figures/wider_dnn_results/}{pearson_chr03.pdf_tex}}
        \caption{chr3}
    \end{subfigure} \hfill
    \begin{subfigure}{0.45\textwidth}
        \scriptsize
        \resizebox{\textwidth}{!}{
        \import{figures/wider_dnn_results/}{pearson_chr05.pdf_tex}}
        \caption{chr5}
    \end{subfigure}\\[5mm]
    \begin{subfigure}{0.45\textwidth}
        \scriptsize
        \resizebox{\textwidth}{!}{
        \import{figures/wider_dnn_results/}{pearson_chr10.pdf_tex}}
        \caption{chr10}
    \end{subfigure}\hfill
    \begin{subfigure}{0.45\textwidth}
        \scriptsize
        \resizebox{\textwidth}{!}{
        \import{figures/wider_dnn_results/}{pearson_chr19.pdf_tex}}
        \caption{chr19}
    \end{subfigure}\\[3mm]
    \centering
    \begin{subfigure}{0.45\textwidth}
        \scriptsize
        \resizebox{\textwidth}{!}{
        \import{figures/wider_dnn_results/}{pearson_chr21.pdf_tex}}
        \caption{chr21}
    \end{subfigure}
    \caption{Pearson correlations, ``wider'' variant of DNN,  test chromosomes}
    \label{fig:results:widerDNN_pearson}
\end{figure}
\begin{figure}[hbp]
 \centering
 \scriptsize
 \import{figures/wider_dnn_results/}{lossOverEpochs.pdf_tex}
 \caption{learning progress for wider DNN} \label{fig:results:widerDNN_lossEpochs}
\end{figure}

The predictions from the ``longer'' variant were partially better for the test set than the initial ones in terms of
Pearson correlations, \cref{fig:results:longerDNN_pearson}.
Interestingly, no predictions were available for certain distances after 250 and 500 epochs, 
while predictions for all distances were available after 1000 epochs.
The reason for this behavior is unknown, but due to the network setup, 
comparatively few neurons are responsible for certain distances.
Since the longer network setup has considerably more trainable parameters,
500 epochs might not be enough to fully adjust the weights of these (outer) neuros.
The learning process in itself looked more smooth and reached a lower validation error than before, \cref{fig:results:longerDNN_lossEpochs}.
\begin{figure}[p]
    \begin{subfigure}{0.45\textwidth}
        \scriptsize
        \resizebox{\textwidth}{!}{
        \import{figures/longer_dnn_results/}{pearson_chr03.pdf_tex}}
        \caption{chr3}
    \end{subfigure} \hfill
    \begin{subfigure}{0.45\textwidth}
        \scriptsize
        \resizebox{\textwidth}{!}{
        \import{figures/longer_dnn_results/}{pearson_chr05.pdf_tex}}
        \caption{chr5}
    \end{subfigure}\\[5mm]
    \begin{subfigure}{0.45\textwidth}
        \scriptsize
        \resizebox{\textwidth}{!}{
        \import{figures/longer_dnn_results/}{pearson_chr10.pdf_tex}}
        \caption{chr10}
    \end{subfigure}\hfill
    \begin{subfigure}{0.45\textwidth}
        \scriptsize
        \resizebox{\textwidth}{!}{
        \import{figures/longer_dnn_results/}{pearson_chr19.pdf_tex}}
        \caption{chr19}
    \end{subfigure}\\[3mm]
    \centering
    \begin{subfigure}{0.45\textwidth}
        \scriptsize
        \resizebox{\textwidth}{!}{
        \import{figures/longer_dnn_results/}{pearson_chr21.pdf_tex}}
        \caption{chr21}
    \end{subfigure}
    \caption{Pearson correlations, ``longer'' variant of DNN,  test chromosomes}
    \label{fig:results:longerDNN_pearson}
\end{figure}
\begin{figure}[hbp]
 \centering
 \scriptsize
 \import{figures/longer_dnn_results/}{lossOverEpochs.pdf_tex}
 \caption{learning progress for longer DNN} \label{fig:results:longerDNN_lossEpochs}
\end{figure}

The Pearson correlations for predictions from the ``wider-longer'' variant are shown in \cref{fig:results:wider-longerDNN_pearson}.
While improvements can be seen for 3 of 5 test chromosomes compared to the initial network, 
the results were worse than the ones from the highly similar ``longer''-variant alone,
and the remaining validation error was also higher.
\begin{figure}[p]
    \begin{subfigure}{0.45\textwidth}
        \scriptsize
        \resizebox{\textwidth}{!}{
        \import{figures/wider-longer_dnn_results/}{pearson_chr03.pdf_tex}}
        \caption{chr3}
    \end{subfigure} \hfill
    \begin{subfigure}{0.45\textwidth}
        \scriptsize
        \resizebox{\textwidth}{!}{
        \import{figures/wider-longer_dnn_results/}{pearson_chr05.pdf_tex}}
        \caption{chr5}
    \end{subfigure}\\[5mm]
    \begin{subfigure}{0.45\textwidth}
        \scriptsize
        \resizebox{\textwidth}{!}{
        \import{figures/wider-longer_dnn_results/}{pearson_chr10.pdf_tex}}
        \caption{chr10}
    \end{subfigure}\hfill
    \begin{subfigure}{0.45\textwidth}
        \scriptsize
        \resizebox{\textwidth}{!}{
        \import{figures/wider-longer_dnn_results/}{pearson_chr19.pdf_tex}}
        \caption{chr19}
    \end{subfigure}\\[3mm]
    \centering
    \begin{subfigure}{0.45\textwidth}
        \scriptsize
        \resizebox{\textwidth}{!}{
        \import{figures/wider-longer_dnn_results/}{pearson_chr21.pdf_tex}}
        \caption{chr21}
    \end{subfigure}
    \caption{Pearson correlations, ``wider-longer'' variant of DNN,  test chromosomes}
    \label{fig:results:wider-longerDNN_pearson}
\end{figure}
\begin{figure}[hbp]
 \centering
 \scriptsize
 \import{figures/wider-longer_dnn_results/}{lossOverEpochs.pdf_tex}
 \caption{learning progress for ``wider-longer'' DNN} \label{fig:results:wider-longerDNN_lossEpochs}
\end{figure}

\subsubsection{Results for variations of the loss function} \label{sec:results:loss_functions}

While the influences of the MSE- and VGG-loss-weights were hard to estimate, 
it was found that the TV loss weight needed to be much smaller than the two other weights.
Otherwise, many true interactions outside the first few matrix diagonals were considered as noise and optimized away early in the training process.
\xxx maybe put one figure here from \texttt{2020-12-05\_tvLoss}

\subsubsection{Results for different binsizes and windowsizes}

\subsection{Hi-cGAN approaches}
\subsubsection{Initial results}
The results at windowsize 64 bins were encouraging \xxx
\begin{figure}[p]
    \begin{subfigure}{0.45\textwidth}
        \scriptsize
        \resizebox{\textwidth}{!}{
        \import{figures/GAN_64/}{pearson_chr03.pdf_tex}}
        \caption{chr3}
    \end{subfigure} \hfill
    \begin{subfigure}{0.45\textwidth}
        \scriptsize
        \resizebox{\textwidth}{!}{
        \import{figures/GAN_64/}{pearson_chr05.pdf_tex}}
        \caption{chr5}
    \end{subfigure}\\[5mm]
    \begin{subfigure}{0.45\textwidth}
        \scriptsize
        \resizebox{\textwidth}{!}{
        \import{figures/GAN_64/}{pearson_chr10.pdf_tex}}
        \caption{chr10}
    \end{subfigure}\hfill
    \begin{subfigure}{0.45\textwidth}
        \scriptsize
        \resizebox{\textwidth}{!}{
        \import{figures/GAN_64/}{pearson_chr19.pdf_tex}}
        \caption{chr19}
    \end{subfigure}\\[3mm]
    \centering
    \begin{subfigure}{0.45\textwidth}
        \scriptsize
        \resizebox{\textwidth}{!}{
        \import{figures/GAN_64/}{pearson_chr21.pdf_tex}}
        \caption{chr21}
    \end{subfigure}
    \caption{Pearson correlations, GAN, windowsize 64, test chromosomes}
    \label{fig:results:GAN64_pearson}
\end{figure}
\begin{figure}[hbp]
 \centering
 \scriptsize
 \import{figures/GAN_64/}{lossOverEpochs.pdf_tex}
 \caption{learning progress cGAN, windowsize 64} \label{fig:results:GAN64_lossEpochs}
\end{figure}
