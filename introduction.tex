\section{Introduction}
In recent years, the three-dimensional organization of DNA has been shown to 
be a key factor for important processes in molecular biology \cite{Chathoth2019,Zhang2019b}.
However, even with the most advanced experimental methods like Hi-C and its derivatives \cite{LiebermanAiden2009, Rao2014, Belaghzal2017}, 
it remains comparatively expensive to study the spatial folding of DNA,
so that current knowledge of three-dimensional DNA organization is still sketchy.
In the last five years, several methods have thus been proposed to improve on this situation
by determining DNA-DNA interactions \emph{in-silico}, using existing experimental data.

However, most current in-silico approaches leave room for improvement, for example with regard to usability, the necessary type and amount of data or the pre- and postprocessing steps required.
The goal of this master thesis is thus to provide an approach for the prediction of Hi-C interaction matrices which is easy to operate and requires only data from \acrshort{cs}
experiments with minimal pre- and postprocessing.

\subsection{Hi-C process and contact matrices} \label{sec:intro:hic}
\begin{wrapfigure}[27]{r}{0.3\textwidth}
 \vspace{-19mm}
 \resizebox{0.3\textwidth}{!}{
 \small
 \import{figures/}{HiC-lab.pdf_tex}}
 \caption{Hi-C lab process}
 \label{fig:intro:HiC}
\end{wrapfigure}
The Hi-C process is an elaborate biochemical procedure for investigating the 
spatial structure of DNA by detecting DNA-DNA interactions within and 
across chromosomes.
The original Hi-C workflow has been developed by Lieberman-Aiden et al. in 2009  \cite{LiebermanAiden2009}
and is depicted in simplified form in \cref{fig:intro:HiC}.

The typical input (In) to Hi-C consists of several millions of cells,
which are treated chemically to fix existing DNA-DNA contacts, 
commoly using formaldehyde, before they are lysed.
Next, the DNA is extracted and cut into fragments by certain restriction enzymes (1),
usually HindIII or DpnII, 
and the cut ends are repaired with nucleotides, some of which are marked by biotin (2).
The free ends are then joined (3) under conditions which prefer
ligations among open ends over ligations between different fragments.
Originally, such conditions were achieved by high dilution of the fragments in
solvents, but especially this part of the protocol has been replaced by 
more efficient methods in later works \cite{Rao2014,Belaghzal2017}.
The ligated fragments are then purified and cut into shorter sequences,
some of which contain biotinylated nucleotides and some not (4).
The fragments of interest -- the ones containing biotinylated nucleotides -- 
are then selected by pulling down biotin (5), for example using magnetic tags,  
and subjected to paired-end DNA-sequencing (6).
In the end, the output of the Hi-C lab process is a large number of short genomic ``reads'',
which are subsequently processed in the bioinformatics part of the Hi-C protocol 
outlined in the following section.

On the software side of the protocol, the reads first need to be mapped 
to the corresponding reference genome.
Here, only reads are kept where the ``left'' sequence (6)(a)
uniquely maps to a different region of the reference genome than the ``right'' sequence (6)(b).
These so-called chimeric reads are subjected to quality control, and those passing are counted as an interaction
between the two genomic positions (1)(a) and (1)(b) to which the two ends belong.
However, at reasonable read coverages, interactions cannot be counted per base pair. 
Instead, the reference genome is split into equally sized bins (or regions), 
and the reads are counted for those regions where they belong. 
The final outcome of a Hi-C experiment is then a sparse matrix, henceforth referred to as ``Hi-C matrix'', 
which records the interaction count for all possible pairs of regions in the reference genome.

In the bioinformatics part of the Hi-C protocol, often just a small fraction of all reads 
fulfill the selection criteria outlined above, for example due to reads not being chimeric or uniquely mappable.
This makes Hi-C a comparatively inefficient, slow and thus expensive process.
For example, creating the well-known dataset by Rao et al. \cite{Rao2014} with matrix bin sizes down to \SI{1}{\kilo\bp}, required several \emph{billions} of reads being made. 
Throughout this thesis, matrix resolutions of \SI{25000}{\bp} are used, unless otherwise noted.

Parts of \cref{fig:intro:HiC} and the process description above 
have been adapted from the preceding study project \cite{Krauth2020}.

\xxx mod input steps in graph
\xxx mark regions a, b throughout the graph

\subsection{Goal of the thesis}
\begin{wraptable}{r}{5.5cm}
\vspace{-10mm}
\centering
\small
\begin{tabular}{lcccc}
cell line & \rotatebox[origin=l]{90}{TF ChIP-seq} & \rotatebox[origin=l]{90}{Histone ChIP-seq} & \rotatebox[origin=l]{90}{Dnase-seq} & \rotatebox[origin=l]{90}{Hi-C} \\ \hline
K562     & 680 & 20 & 10 & 1 \\
HepG2    & 668 & 15 & 3  & 1 \\
A549     & 251 & 87 & 14 & 6 \\
HEK293   & 231 & 6  &    &   \\
GM12878  & 211 & 15 & 3  & 7 \\
MCF-7    & 155 & 18 & 8  &   \\
H1       & 88  & 55 & 3  &   \\
HeLa-S3  & 77  & 15 & 4  & 1 \\
SK-N-SH  & 62  & 21 & 2  &   \\
IMR-90   & 16  & 34 & 2  & 2 \\
HCT116   & 27  & 17 & 1  & 8 \\
H9       &     & 35 & 7  &   \\
GM23338  & 15  & 13 & 1  &   \\
Ishikawa & 25  &    & 4  &   \\
HEK293T  & 17  &    & 1  &   \\
GM23248  & 2   & 13 & 1  & 1 \\ \hline
\end{tabular}
\caption{availability of selected assays in ENCODE (extract)} \label{tab:intro:assaysInEncode}
\vspace{-10mm}
\end{wraptable}
Due to the high effort, Hi-C experiments have been conducted only by few labs and for comparatively few organisms and cell lines so far.
For example, as of January 2021, Hi-C assays are available for just 26 of more than 150 human cell lines listed in the Encyclopedia of DNA elements \cite{Encode2012,Davis2017},
and these have been produced by only five university labs.

However, investigations and experiments with the available Hi-C data have shown correlations
between the spatial DNA structure and the binding sites of certain transcription factors and histone modifications \cite{Rao2014, Bonev2016}.
Detecting these binding sites experimentally is less involved than the Hi-C process and can be done by the well-established ChIP-seq method \cite{Johnson2007,Robertson2007}.
\Cref{tab:intro:assaysInEncode} to the side shows the first 16 human cell lines in ENCODE, sorted by total number of selected assays available.
It is obvious that even otherwise well-investigated cell lines like HEK293 and MCF-7 often lack Hi-C data, while ChIP-seq data is mostly available.

The goal of this master thesis is thus to predict missing Hi-C data from existing ChIP-seq data,
making use of the known correlations between the binding sites of transcription factors and histone modifications of the one hand 
and Hi-C interaction counts on the other hand. 
Since this is a wide field and lead time for a master thesis is limited, 
the present work will focus on machine learning techniques,
which have proven a good choice for exploiting complex patterns
and relationships in various kinds of data. \xxx citations needed
More specifically, the goal of this thesis is to develop an easy-to-use machine learning
approach for predicting Hi-C interaction matrices from ChIP-seq data, 
using standard in- and output formats with minimal pre- and postprocessing.

\subsection{Structure of the thesis}
