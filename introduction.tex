\section{Introduction}
In recent years, the three-dimensional organization of DNA has been shown to 
be a key factor for important processes in molecular biology \cite{Chathoth2019,Zhang2019b}.
However, even with the most advanced experimental methods like Hi-C and its derivatives \cite{LiebermanAiden2009, Rao2014, Belaghzal2017}, 
it remains comparatively expensive to study the spatial folding of DNA,
so that current knowledge of three-dimensional DNA organization is still sketchy.
In the last five years, several methods have thus been proposed to improve on this situation
by determining DNA-DNA interactions \emph{in-silico}, using existing experimental data.

However, most current in-silico approaches leave room for improvement, for example with regard to usability, the necessary type and amount of data or the pre- and postprocessing steps required.
The goal of this master thesis is thus to provide an approach for the prediction of Hi-C interaction matrices which is easy to operate and requires only data from \acrshort{cs}
experiments with minimal pre- and postprocessing.

\subsection{Hi-C process and contact matrices} \label{sec:intro:hic}
\begin{wrapfigure}[36]{r}{0.3\textwidth}
 \resizebox{0.3\textwidth}{!}{
 \small
 \import{figures/}{HiC-lab.pdf_tex}}
 \caption{Hi-C lab process}
 \label{fig:intro:HiC}
\end{wrapfigure}
The Hi-C process is an elaborate biochemical procedure for investigating the 
spatial structure of DNA by detecting DNA-DNA interactions within and 
across chromosomes.
The Hi-C workflow has originally been developed by Lieberman-Aiden et al. in 2009  \cite{LiebermanAiden2009}
and is depicted in simplified form in \cref{fig:intro:HiC} \cite{Krauth2020}.

\subsection{Goal of the thesis}
\subsection{Structure of the thesis}
