\section{Discussion and Outlook}
In this thesis, two methods have been considered for predicting Hi-C interaction matrices,
one based on a dense neural network proposed by Farr\'e et al. \cite{Farre2018a} and the other one based on a novel conditional generative adversarial network
inspired by \emph{pix2pix} \cite{Isola2017}.

Despite all amendments to the original network setup,
the results of the dense neural network approach remained modest.
Improvements, if at all visible, could only be achieved for a small part of the test set.

On the other hand, the novel conditional generative adversarial network method seems promising.
Especially the variant with CNN embedding showed widely satisfying results for genomic distances up to about \SI{5}{\mega\bp}
and partially outperformed existing methods like \emph{HiC-Reg} \cite{Zhang2019} with regard to predicting nested structures.
However, there is still room for improvement -- even the best predictions of this thesis were not or not much better 
than simply taking data from the training cell line as a ``prediction'' for the target cell line.

Throughout the thesis, at least two starting points for further improving the predictions were identified.
As of now, the discriminator is trained on two kinds of inputs:
it is supposed to classify pairs of true chromatin feature arrays and true Hi-C submatrices as real, 
and pairs of true chromatin feature arrays and artificial Hi-C submatrices produced by the generator as fake, cf. \cref{eq:improve:disc_loss_total}.
For text-to-image synthesis tasks, a third kind of input has been introduced; here the discriminator is additionally trained 
to classify pairs of true output and mismatching \emph{input data} as fake \cite{Reed2016}.
Training the discriminator on such samples might help the \acrshort{cgan} to avoid situations 
where the generator produces matrices which ``look good'', 
but have insufficient correlations with the chromatin feature input data.
It is unclear how strong the effect of this change would be in the given application, 
where the adversarial loss is only one part of the combined generator loss, but certainly worth a try.

As noted in \cref{sec:methods:cGAN_initial}, better results were obtained when using sigmoid activation for the generator,
replacing the tanh activation of the original \emph{pix2pix} network.
Retrospectively, however, it might have been better to leave the tanh activation in the generator untouched
and instead scale the training matrices -- and maybe the chromatin features, too -- to value range -1...1.
This would allow for a broader input value range to the sigmoid function contained in the discriminator loss,
potentially improving numerical gradient computations.

\xxx \xxx \xxx
Better embedding network, choice was ad-hoc
\xxx \xxx \xxx
