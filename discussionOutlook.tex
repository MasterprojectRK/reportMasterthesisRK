\section{Discussion and Outlook}
In this thesis, two methods have been considered for predicting Hi-C interaction matrices,
one based on a \acrlong{dnn} proposed by Farr\'e et al. \cite{Farre2018a} and the other one based on a \acrlong{cgan}
inspired by \emph{pix2pix} \cite{Isola2017}, an image transformation network.

Despite all amendments to the original network setup,
the results of the \acrlong{dnn} approach remained modest.
Improvements, if at all visible, could only be achieved for a small part of the test set.
However, the comparison with the results from Farr\'e et al. \cite{Farre2018a} -- who have used the same network with different kinds of input data --
shows that the failure of the \acrshort{dnn} approach is likely due to the chosen way of sample preparation 
and not necessarily founded in an unsuitability of the network setup itself.
Retrospectively, it would obviously have been better to test the \acrshort{dnn} in its original setup first and reproduce the results from Farr\'e et al. \cite{Farre2018a} 
before porting it to different organisms with different cell lines and chromosome folding principles.

Independent from the \acrshort{dnn}, the newly developed \acrlong{cgan} method, Hi-cGAN, seems promising.
Especially the variant with \acrshort{cnn} embedding showed widely satisfying results for genomic distances up to about \SI{5}{\mega\bp}
and partially outperformed existing methods like \emph{HiC-Reg} \cite{Zhang2019}, especially with regard to predicting nested interacting regions.
Interestingly, the approach worked not only for human cell lines like GM12878 and K562, where comparatively many training samples are available,
but also for Drosophila melanogaster embryos, where much fewer samples can be extracted from the relatively short genome.
However, there is still room for improvement -- even the best predictions within this thesis were not or not much better 
than simply taking data from the training cell line as a ``prediction'' for the target cell line.

Throughout the thesis, a few starting points for further improving Hi-cGAN's predictions were identified.
In the current implementation, the discriminator is trained on two kinds of inputs:
it is supposed to classify pairs of true chromatin feature arrays and true Hi-C submatrices as real, 
and pairs of true chromatin feature arrays and artificial Hi-C submatrices produced by the generator as fake, cf. \cref{eq:improve:disc_loss_total}.
For text-to-image synthesis tasks, a third kind of input has been introduced; here the discriminator is additionally trained 
to classify pairs of true output and  \emph{mismatching input data} as fake \cite{Reed2016}.
Training the discriminator on such samples might help the \acrshort{cgan} to avoid situations 
where the generator produces matrices which ``look real'', 
but have insufficient correlations with the chromatin feature input data.
It is unclear how strong the effect of this change would be in the given application, 
where the adversarial loss is only one part of the combined generator loss, but certainly worth a try.

As noted in \cref{sec:methods:cGAN_initial}, better results were obtained by replacing the tanh activation in the generator 
of the original \emph{pix2pix} network by sigmoid activation.
Retrospectively, however, it might have been better to leave the tanh activation in the generator untouched
and instead scale the training matrices -- and maybe the chromatin features, too -- to value range -1...1.
This would allow for a broader input value range to the sigmoid function contained in the discriminator loss,
potentially improving numerical gradient computations.

Another starting point is for sure the network architecture in itself.
While good reasons backed the decision for choosing \emph{pix2pix} for the general network setup,
namely its wide range of applications and demonstrated convergence even when only few training samples are at hand \cite{Isola2017},
it remains a network designed for converting \emph{input images} to output images.
However, the task at hand essentially requires transforming \emph{one-dimensional input} data into very special symmetric output images.
Although surprisingly good results were obtained, it is difficult to imagine that \emph{pix2pix} -- amended by the simplistic and rather ad-hoc \acrshort{cnn} embedding network 
described in \cref{sec:improve:CNN_embedding} and \ref{sec:methods:cnn-embedding} -- is the optimal \acrshort{cgan} for suchlike 1D\,$\rightarrow$\,2D conversions.

Nevertheless, the \acrshort{cgan} approach developed within this thesis can hopefully serve as a helpful starting point and baseline
for further development of improved Hi-C contact matrix prediction methods.
To this end, all source code is provided on github under an open source license \cite{Krauth2021a, Krauth2021b}.
\clearpage
